%% ----------------------------------------------------------------
%% Random bits of latex packages and settings that I've used at some point
%% ---------------------------------------------------------------- 

% Set up the document
\documentclass[a4paper, 11pt, oneside,table]{Thesis}  % Use the "Thesis" style, based on the ECS Thesis style by Steve Gunn
\graphicspath{{Figures/}}  % Location of the graphics files (set up for graphics to be in PDF format)

%Use natbib packagefor bibliography
\usepackage[round, comma, sort&compress, authoryear]{natbib} %Round brackets, comma separator, author-year citation style

%For sideways images
\usepackage{rotating}

%For double hline in tables
\usepackage{hhline}

%Allows images to spill over the edges of the page
\usepackage[export]{adjustbox}

%Use for line split in equations
\usepackage{amsmath}

%Colours for cells in tables - if it gives you an error using [tables] option then include 'tables' in your documentclass
\usepackage[usenames,dvipsnames]{xcolor}% http://ctan.org/pkg/xcolor

%Right adjusted columns in tables
\usepackage{array,booktabs,ragged2e}
\newcolumntype{R}[1]{>{\RaggedLeft\arraybackslash}p{#1}}

\usepackage{verbatim}  % Needed for the "comment" environment to make LaTeX comments
\usepackage{vector}  % Allows "\bvec{}" and "\buvec{}" for "blackboard" style bold vectors in maths

\PassOptionsToPackage{hyphens}{url}\usepackage{hyperref}%Allows the use of hyphens option within url under the hyperref package

\hypersetup{citecolor=black,
	filecolor=black,
	linkcolor=black,
	urlcolor=black,
	colorlinks=true}  % Colours hyperlinks in blue, but this can be distracting if there are many links.

\makeatletter
\g@addto@macro{\UrlBreaks}{\UrlOrds}%Allows special characters in urls to work
\makeatother

\newcommand{\HRule}{\rule{\linewidth}{0.75mm}} % New command to make the lines in the title page

%For Code Insertion
\usepackage{listings}
\definecolor{dkgreen}{rgb}{0,0.6,0}
\definecolor{gray}{rgb}{0.5,0.5,0.5}
\definecolor{mauve}{rgb}{0.58,0,0.82}
\lstset{frame=tb,
  language=C++,		%Change language to whichever you need
  aboveskip=3mm,
  belowskip=3mm,
  showstringspaces=false,
  columns=flexible,
  basicstyle={\small\ttfamily},
  numbers=none,
  numberstyle=\tiny\color{gray},
  keywordstyle=\color{blue},
  commentstyle=\color{dkgreen},
  stringstyle=\color{mauve},
  breaklines=true,
  breakatwhitespace=true,
  tabsize=3
}

\setstretch{1.3}  %Set linespacing - It is better to have smaller font and larger line spacing than the other way round

% Define the page headers using the FancyHdr package and set up for one-sided printing
\fancyhead{}  % Clears all page headers and footers
\rhead{\thepage}  % Sets the right side header to show the page number
\lhead{}  % Clears the left side page header

\pagestyle{fancy}  % Finally, use the "fancy" page style to implement the FancyHdr headers

%% ----------------------------------------------------------------
\frontmatter
\label{Bibliography}
\lhead{\emph{Bibliography}}  % Change the left side page header to "Bibliography"
\bibliographystyle{agsm} % Use agsm to alphabetically order the bibliography - harvard referencing style based
\bibliography{Bibliography}  % The references (bibliography) information are stored in the file named "Bibliography.bib"

\end{document}  % The End
%% ----------------------------------------------------------------